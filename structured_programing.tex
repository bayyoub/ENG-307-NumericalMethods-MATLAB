% This LaTeX was auto-generated from MATLAB code.
% To make changes, update the MATLAB code and export to LaTeX again.

\documentclass{article}

\usepackage[utf8]{inputenc}
\usepackage[T1]{fontenc}
\usepackage{lmodern}
\usepackage{graphicx}
\usepackage{color}
\usepackage{hyperref}
\usepackage{amsmath}
\usepackage{amsfonts}
\usepackage{epstopdf}
\usepackage[table]{xcolor}
\usepackage{matlab}
\usepackage[paperheight=795pt,paperwidth=614pt,top=72pt,bottom=72pt,right=72pt,left=72pt,heightrounded]{geometry}

\sloppy
\epstopdfsetup{outdir=./}
\graphicspath{ {./structured_programing_media/} }

\begin{document}

\matlabtitle{\textbf{Structure Programming}}

\begin{par}
\begin{flushleft}
Here, we will go over specific examples for each the different types of structures programming. 
\end{flushleft}
\end{par}

\matlabheading{If Statements}

\begin{par}
\begin{flushleft}
If statements allow you to execute a set of statements if a logical conditions is true. First we need to define operators:
\end{flushleft}
\end{par}

\begin{par}
\begin{flushleft}
\includegraphics[width=\maxwidth{41.04365278474661em}]{image_0}
\end{flushleft}
\end{par}

\begin{par}
\begin{flushleft}
let's try an example:
\end{flushleft}
\end{par}

\begin{matlabcode}
x = rand()

if x > 0.5 %You start with if, then condition.
    fprintf('Yes! x is greater than 0.5') %Press enter, and MATLAB will automatically indent, if not press tab.
end % Finally, type in end, and MATLAB will automatically indent backwards.
\end{matlabcode}


\begin{par}
\begin{flushleft}
Here, we chose the greater than operation, however, we could've used any other logical operation.
\end{flushleft}
\end{par}

\begin{matlabcode}
x = round(rand()) %This will randomly generate 0 or 1

if x >= 0 %You start with if, then condition.
    error('zero value encountered') %Press enter, and MATLAB will automatically indent, if not press tab.
end % Finally, type in end, and MATLAB will automatically indent backwards.

%Here, I used the error function, which basically displays your desired
%error message.
\end{matlabcode}


\matlabheading{Logical Operators}

\begin{par}
\begin{flushleft}
In MATLAB, we can also utilize logical operators, which perform element-wise comparisons. In MATLAB, if a logical condition is \underline{\textbf{true}}, MATLAB will return the value \underline{\textbf{1}}, if it is \underline{\textbf{false}} it will return the value \underline{\textbf{0}}.
\end{flushleft}
\end{par}

\begin{par}
\begin{flushleft}
\includegraphics[width=\maxwidth{52.282990466633215em}]{image_1}
\end{flushleft}
\end{par}

\begin{par}
\begin{flushleft}
For example:
\end{flushleft}
\end{par}

\begin{matlabcode}
% Generate two random numbers between -1 and 1

a = 2*rand()-1
\end{matlabcode}
\begin{matlaboutput}
a = -0.2369
\end{matlaboutput}
\begin{matlabcode}
b = 2*rand()-1
\end{matlabcode}
\begin{matlaboutput}
b = 0.5310
\end{matlaboutput}
\begin{matlabcode}

% And Statement

a > 0 & b > 0
\end{matlabcode}
\begin{matlaboutput}
ans = logical
   0

\end{matlaboutput}
\begin{matlabcode}

% Or Statement

a > 0 | b > 0
\end{matlabcode}
\begin{matlaboutput}
ans = logical
   1

\end{matlaboutput}
\begin{matlabcode}

% Not Statement 

a ~= b
\end{matlabcode}
\begin{matlaboutput}
ans = logical
   1

\end{matlaboutput}


\matlabheadingtwo{Logical Expressions}

\begin{par}
\begin{flushleft}
Now, what would happen if we have multiple logical operators in a single expression. How does MATLAB evaluate such expression?
\end{flushleft}
\end{par}

\begin{enumerate}
\setlength{\itemsep}{-1ex}
   \item{\begin{flushleft} The first thing that MATLAB does is to evaluate any mathematical expression. \end{flushleft}}
   \item{\begin{flushleft} Next, MATLAB evaluates all relational expressions/operations \end{flushleft}}
   \item{\begin{flushleft} Logical operators are then evaluated in priority order, where \textasciitilde{} has the highest priority, \& second, and | has the lowest priority. (\textasciitilde{} \textgreater{} \& \textgreater{} |) \end{flushleft}}
   \item{\begin{flushleft} If you have two operators that are the same in a row, the left-to-right rule applies. \end{flushleft}}
\end{enumerate}

\begin{par}
\begin{flushleft}
Let's take this logical expression and solve it step by step:
\end{flushleft}
\end{par}

\begin{par}
\begin{flushleft}
a * b \textgreater{} 0 \& b == 2 \& x \textgreater{} 7 | \textasciitilde{}(y \textgreater{} 'd')
\end{flushleft}
\end{par}

\begin{par}
\begin{flushleft}
where a = -1, b = 2, x = 1 and y = 'b'.
\end{flushleft}
\end{par}

\begin{matlabcode}
a = -1; b =2; x = 1; y = 'b';
\end{matlabcode}

\begin{par}
\begin{flushleft}
For convenience, substitute:
\end{flushleft}
\end{par}

\begin{par}
\begin{flushleft}
-1 * 2 \textgreater{} 0 \& 2 == 2 \& 1 \textgreater{} 7 | \textasciitilde{}( 'b' \textgreater{} 'd' )
\end{flushleft}
\end{par}

\begin{par}
\begin{flushleft}
Now that we substituted all the numbers and values in, we evaluate all mathematical expressions:
\end{flushleft}
\end{par}

\begin{par}
\begin{flushleft}
-2 \textgreater{} 0 \& 2 == 2 \& 1 \textgreater{} 7 | \textasciitilde{}( 'b' \textgreater{} 'd')
\end{flushleft}
\end{par}

\begin{par}
\begin{flushleft}
Now evaluate all relational expressions:
\end{flushleft}
\end{par}

\begin{matlabcode}
-2 > 0
\end{matlabcode}
\begin{matlaboutput}
ans = logical
   0

\end{matlaboutput}
\begin{matlabcode}
2 == 2
\end{matlabcode}
\begin{matlaboutput}
ans = logical
   1

\end{matlaboutput}
\begin{matlabcode}
1 > 7
\end{matlabcode}
\begin{matlaboutput}
ans = logical
   0

\end{matlaboutput}
\begin{matlabcode}
'b' > 'd'
\end{matlabcode}
\begin{matlaboutput}
ans = logical
   0

\end{matlaboutput}


\vspace{1em}
\begin{par}
\begin{flushleft}
This means that we have :
\end{flushleft}
\end{par}

\begin{par}
\begin{flushleft}
F \& T \& F | \textasciitilde{} F
\end{flushleft}
\end{par}

\begin{par}
\begin{flushleft}
From earlier \textasciitilde{} has the highest priority. Remember \textasciitilde{} returns the opposite of what you have therefore \textasciitilde{}F --\textgreater{} T:
\end{flushleft}
\end{par}

\begin{par}
\begin{flushleft}
F \& T \& F | T
\end{flushleft}
\end{par}

\begin{par}
\begin{flushleft}
Next evaluate the \& operator. Since we have two of them, start from left, and proceed right:
\end{flushleft}
\end{par}

\begin{par}
\begin{flushleft}
Take F \& T \& F and split it into two segments: 
\end{flushleft}
\end{par}

\begin{par}
\begin{flushleft}
F \& T = F
\end{flushleft}
\end{par}

\begin{par}
\begin{flushleft}
Now you have:
\end{flushleft}
\end{par}

\begin{par}
\begin{flushleft}
F \& F | T
\end{flushleft}
\end{par}

\begin{par}
\begin{flushleft}
F | T 
\end{flushleft}
\end{par}

\begin{par}
\begin{flushleft}
The final answer is T. 
\end{flushleft}
\end{par}


\matlabheading{If and elseif Statements}

\begin{par}
\begin{flushleft}
Now that we have more sophisticated tool, let's consider other Decision problems:
\end{flushleft}
\end{par}

\end{document}
